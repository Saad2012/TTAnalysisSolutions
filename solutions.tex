\documentclass{article}
\usepackage{amsmath}

\title{Solution Manual for Analysis I}
\author{Dean F. Valentine Jr.}

\begin{document}
    \maketitle
    \section{Beginning Chapter}
    \section{Natural Numbers}
    \section{Set Theory}
    \subsection{Fundamentals}
    \begin{enumerate}
        \item Reflexivity:
            \begin{enumerate}
                \item Equality between sets is defined as: $(\forall x \in A: x
                    \in B) \land (\forall x \in B: x \in A)$
                \item A set is defined as a collection of objects.
                \item It is a contradiction for any object $x$ to satisfy $x
                    \in A$, but also $x \notin A$, so equality of sets
                    is reflexive ($A = A$), meaning all objects that satisfy
                    the claim $x \in A$ also satisfy $x \in A$, and
                    ``vice-versa''
            \end{enumerate}
            Symmetry:
            \begin{enumerate}
                \item To say that sets A and B are equal, as per our definition
                    of equality, is to say $\forall x \in A: x \in B \land
                    \forall x \in B: x \in A$.
                \item So, if $A = B$, and by definition $(\forall x \in A: x \in
                    B) \land (\forall x \in B: x \in A)$, then we can just reorder
                    the given claims around the (commutative) ``and'',show that $(\forall x \in B: x
                    \in A) \land (\forall x \in A: x \in B)$, and see that B = A.
            \end{enumerate}
            Transitivity:
            \begin{enumerate}
                \item Suppose that among three sets A,B, and C, sets A = B, and
                    B = C
                \item Consider $\exists x : x \in A$
                \item Since $A = B$, $\forall y \in A : y \in B$, and so $x \in B$
                \item But also because $B = C$, $\forall z \in B, z \in C$, and so also $x \in C$
                \item Thus, $\forall x \in A, x \in C$, and the same logic can
                    be applied in reverse to find $\forall x \in C, x \in A$, and so $A = C$
            \end{enumerate}
        \item
            \begin{enumerate}
                \item Axiom 3.2 says that there exists a set $\emptyset$, the set of no elements.
                \item Axiom 3.1 says that if A is a set, it is also an object,
                    and by the definition of sets (3.1.1), there exists the set $X$
                    such that $A \in X$. Thus, there exists some set $A_1$ which contains the object $\emptyset$
                \item Two sets are not equal if they do not share each others
                    objects. $A_1$ contains $\emptyset$, but $\emptyset$ is
                    defined as having lacking objects, so these two sets are
                    not equal.
                \item Similarly we can consider the set $A_2$ which contains only $A_1$, which is not equal to
                    either $A_1$ or $\emptyset$ because neither of them contain $A_1$, and instead contain
                    $\emptyset$ and no objects, respectively.
                \item Lastly we think of a set $A_3 : \emptyset \in A_3 \land
                    A_2 \in A_3$, and see that this does not contain the same
                    objects as $\emptyset$, $A_1$, or $A_3$. $\emptyset \notin
                    A_2$, $A_1 \notin A_1$, and neither $\emptyset$ nor $A_1$
                    are in $\emptyset$ so at this point we have proven 3.2.
            \end{enumerate}
        \item Proof $\{a,b\} = \{a\} \cup \{b\}$:
            \begin{enumerate}
                \item The union between sets A and B as defined in
                    Axiom 3.4 is the set whose elements are contained by A
                    or B or both.
                \item $\{a\} \cup \{b\}$ contains $a$, because $a \in \{a\}$
                \item $\{a\} \cup \{b\}$ contains $b$, because $b \in \{b\}$
                \item There can be no other objects in  $\{a\} \cup \{b\}$, because 
                    all objects in $\{a\} \cup \{b\}$ must either be in $\{a\}$
                    or $\{b\}$ or both by the definition of a union, and the
                    existence of a third object would imply that $\{a\}$ and
                    $\{b\}$ contained altogether three distinct objects, a contradiction.
            \end{enumerate}
            Proof of commutativity:
            \begin{enumerate}
                \item Suppose $\exists x \in A \cup B$. 
                \item That means by Axiom 3.4 $x \in A \lor x \in B$. 
                \item If $x \in A$, then $x \in B \cup A$, because $B \cup A$
                    is the set of all elements either in $B$ or $A$.  
                \item If $x \in B$, then by the same logic $x \in B \cup A$ as
                    well, and so we can say $\forall x \in A \cup B: x \in B
                    \cup A$, and we have proven their equality by the
                    definition of set equality given earlier.
            \end{enumerate}
            Proof $A \cup A = A$
            \begin{enumerate}
                \item Suppose for the sake of contradiction, the hypothesis $\exists x \in A: x \notin A \cup A$.
                \item By the definition of union, $x \notin A \cup A
                    \rightarrow x \notin A \land x \notin A$, logically
                    equivalent to $x \notin A \cup A \rightarrow x \notin A$,
                    a contradiction. So, $\forall x \in A: x
                    \in A \cup A$. 
                \item Suppose for the sake of contradiction, the hypothesis $\exists x \in A \cup A: x \notin A$.
                \item But $x \in A \cup A \implies (x \in A \lor x \in A)$,
                    which is equivalent to saying $x \in A \cup A \implies (x
                    \in A)$, a contradiction. So we know that $\forall x \in A
                    \cup A: x \in A$, and that $A$ must equal $A \cup A$ by the
                    definition of set equality.
            \end{enumerate}
            Proof $A \cup \emptyset = A$ and $\emptyset \cup A = A$
            \begin{enumerate}
                \item $\forall x \in A \cup \emptyset: x \in A \lor x \in \emptyset$, by Axiom 3.4
                \item By axiom 3.2, $\neg\exists x \in \emptyset$.
                \item This means that $\forall x \in A \cup \emptyset: x \in
                    A$. Now we must prove $\neg\exists x \in A : x \notin A \cup \emptyset$.
                \item Suppose for sake of contradiction $\exists x \in A : x \notin A \cup \emptyset$.
                \item This would mean $x \notin A \land x \notin \emptyset$, by Axiom 3.4.
                \item $x$ is obviously not in the empty set, but we assumed $x \in A$,
                    and thus we have to have contradicted ourselves.
                \item Since $(\forall x \in A \cup \emptyset: x \in A) \land
                    (\forall x \in A: x \in A \cup \emptyset)$, $A = A \cup
                    \emptyset$.
                \item Because of commutitivity this also means $A = \emptyset \cup A$.
            \end{enumerate}
        \item Proof $A \subseteq B \land B \subseteq C \implies A \subseteq C$:
            \begin{enumerate}
                \item $A \subseteq B \equiv \forall x \in A : x \in B$
                \item $B \subseteq C \equiv \forall x \in B : x \in C$
                \item Since $(\forall x \in A: x \in B) \land (x \in B \implies x \in
                    C)$, $\forall x \in A: x \in C$
            \end{enumerate}
            Proof $A \subseteq B \land B \subseteq A \implies A = B$
            \begin{enumerate}
                \item $A \subseteq B \equiv \forall x \in A : x \in B$
                \item $B \subseteq A \equiv \forall x \in B : x \in A$
                \item $\forall x \in A: x \in B \land \forall x \in B: x \in A
                    \equiv A = B$, so these starting propositions combined are
                    the definition of equality established earlier in the
                    chapter.
            \end{enumerate}
            Proof $A \subset B \land B \subset C \implies A \subset C$:
            \begin{enumerate}
                \item $A \subset C \equiv \forall x \in A: x \in C \land A \neq C$, so we must prove
                    two things: $A \neq C$, and $\forall x \in A: x \in C$
                \item $A \subset B \equiv \forall x \in A: x \in B \land A \neq B$
                \item $A \neq B \implies \exists x \in A: x \notin B \lor \exists x \in B: x \notin A$
                \item Since by the definition of a subset $\neg\exists x \in A:
                    x \notin B$, this and the relation $A \neq B$ implies
                    $\exists x \in B: x \notin A$
                \item $B \subset C \equiv \forall x \in B: x \in C \land B \neq C$
                \item Since $\forall x \in B: x \in C$ and $\exists x \in B: x
                    \notin A$, $\exists x \in C: x \notin A$, and $C \neq A$.
                \item But also, $\forall x \in A: x \in B \land \forall x \in B: x \in C$. This
                    means $x \in A \implies x \in B \implies x \in C$, and we see $\forall x \in A: x \in C$.
                \item These two statements ($A \neq C$, and $\forall x \in A: x \in C$) are equivalent
                    to saying ($A \subset C$), and so we have finished our proof.
            \end{enumerate}
        \item Proof $A \subseteq B \implies A \cup B = B \land A \cap B = B$:
            \begin{enumerate}
                \item $A \subseteq B \equiv \forall x \in A: x \in B$
                \item $A \cup B \equiv \{x \in A \lor x \in B\}$
                \item $\forall x \in A: x \in B$, so $\forall x \in A : x \in
                    {x \in A \lor x \in B}$ and $\forall x \notin A : x \notin
                    {x \in A \lor x \in B}$, proving that if $A \subseteq B
                    \implies A \cup B = B$
                \item $A \cap B \equiv \{x \in A \land x \in B\}$
                \item $A \subseteq B \equiv \forall x \in A: x \in B$, so if $A \subseteq B$, then
                    $\{x \in A \land x \in B\} = \{x \in B\}$
                \item $B \equiv \{x \in B\}$, so $A \subseteq B \implies A \cap
                    B = B \land A \cup B = B$, and we have finished our proof.
            \end{enumerate}
            Proof $A \cup B = B \implies A \subseteq B \land A \cap B = B$:
            \begin{enumerate}
                \item $A \subseteq B \implies A \cap B = B$, as per earlier
                    proofs, so we need only prove $A \cup B = B \implies A
                    \subseteq B$
                \item $A \subseteq B \equiv \forall x \in A: x \in B$
                \item $B = A \cup B \equiv \forall x \in B: x \in A \cup B
                    \land \forall x \in A \cup B : x \in B$
                \item $A \cup B \equiv \{x \in A \lor x \in B\}$, so
                \item $\forall x \in A \lor x \in B : x \in B$ and
                \item $\neg\exists x \in A : x \notin B$, the contrapositive of which is
                \item $\forall x \in A : x \in B$, and so we have proved our hypothesis.
            \end{enumerate}
            Proof $A \cap B = A \implies A \subseteq B \land A \cup B = B$:
            \begin{enumerate}
                \item $A \subseteq B \implies A \cup B = B$, as per earlier
                    proofs, so we need only prove $A \cap B = A \implies A
                    \subseteq B$
                \item $A \subseteq B \equiv \forall x \in A: x \in B$
                \item $A \cap B = A \equiv \{x \in A\} = \{x \in A \land x \in B\}$
                \item Thus, $A \cap B = A \implies \forall x \in A: x \in B$, or
                    $A \cap B = B \implies A \subseteq B$, so we have already proved our
                    hypothesis.
            \end{enumerate}
        \item Let A, B, C be sets, and let X be a set containing A, B, C as subsets. \\
            Proof $A \cup X = X \cap A = X$: See 3.1.5 \\
            Proof $A \cup A = A$:
            \begin{enumerate}
                \item Suppose for sake of contradiction suppose $A \cup A \neq A$.
                \item This would mean either A contains an item not contain in $A \cup A$, or
                    $A \cup A$ contains an item not contained in X.
                \item Suppose $\exists x \in A \cup A : x \notin A$
                \item $x \in A \cup A \equiv x \in A \lor x \in A$, which when
                    simplified means $x \in A$.
                \item We have now already contradicted ourselves, by declaring
                    $x \in A \land x \notin A$.
                \item Suppose then that $\exists x : x \in A \land x \notin A \cup A$
                \item We can just as easily see that $x \notin A \cup A \equiv
                    x \notin A \land x \notin A$, equivalent to saying $x
                    \notin A$.
                \item $x \notin A$ contradicts our earlier assumption that $x \in A$, and our
                    premises must be incorrect, and $\neg\exists x : x \in A \land x \notin A \cup A$.
                \item Thus, $A \cup A$ has the same elements as $A$ and is
                    equivalent by our definition.
            \end{enumerate}
            Proof $A \cap A = A$
            \begin{enumerate}
                \item Suppose for sake of contradiction suppose $A \cap A \neq A$.
                \item This would mean either A contains an item not contain in
                    $A \cap A$, or $A \cap A$ contains an item not contained in
                    X.
                \item Suppose $\exists x : x \in A \cap A \land x \notin A$
                \item This means that $x \in A \land x \in A$, simplified $x
                    \in A$.
                \item We have now already contradicted ourselves, by declaring
                    $x \in A \land x \notin A$.
                \item Suppose then that $\exists x : x \in A \land x \notin A \cap A$
                \item We can just as easily see that $x \notin A \cap A \equiv
                    x \notin A \lor x \notin A$, equivalent to saying $x \notin
                    A$.
                \item $x \notin A$ contradicts our earlier assumption that $x \in A$, and our
                    premises must be incorrect, and $\neg\exists x : x \in A \land x \notin A \cap A$.
                \item Thus, $A \cap A$ has the same elements as $A$ and is
                    equivalent by our definition.
            \end{enumerate}
            Proof of Distributivity ($A \cap (B \cup C) = (A \cup B) \cup C$)
            \begin{enumerate}
                \item 
            \end{enumerate}
        \item Proof $A \cap B \subseteq A \land A \cap B \subseteq B$
            \begin{enumerate}
                \item $A \cap B = B \cap A$ because of commutativity, so we
                    need only prove $A \cap B \subseteq A$.
                \item $(A \cap B) \equiv \{x \in A \land x \in B\}$
                \item Since $x \in {x \in A \land x \in B} \implies x \in A$,
                    $\forall x \in (A \cap B) : x \in A$, which is the definition
                    of $(A \cap B) \subseteq A$.
            \end{enumerate}
            Proof $C \subseteq A \land C \subseteq B \iff C \subseteq A \cap B$
            \begin{enumerate}
                \item This is a bidirectional equality, so we have to prove both that
                    $C \subseteq A \cap B \implies C \subseteq A \land C \subseteq B$ and
                    that $C \subseteq A \land C \subseteq B \implies C \subseteq A \land B$
                \item $C \subseteq A \cap B \equiv \forall x \in C: x \in A \land x \in B$
                \item $\forall x \in C: x \in A \land x \in B \implies \forall x \in C: x \in A$, so
                    $C \subseteq A$. Similarly, $\forall x \in C: x \in A \land x \in B \implies \forall x \in C: x \in B$, so
                    $C \subseteq B$. Thus, $C \subseteq A \cap B \implies C \subseteq A \land C \subseteq B$
                \item $C \subseteq A \land C \subseteq B \equiv \forall x \in C: x \in A \land \forall x \in C: x \in B$
                \item $\forall x \in C: x \in A \land \forall x \in C: x \in B
                    = \forall x \in C: x \in A \land x \in B$, so we also have
                    $\forall x \in C: x \in A \land x \in B$.
                \item $A \cap B$ is the set of elements $\forall x: x \in A
                    \cap B$, so $\forall x \in C: x \in A \cap B$ or $C
                    \subseteq A \cap B$, so $C \subseteq A \land C \subseteq B
                    \implies C \subseteq A \cap B$. This concludes our proof
                    that $C \subseteq A \cap B \iff C \subseteq A \land C
                    \subseteq B$.
            \end{enumerate}
        \item Proof $A \cap (A \cup B) = A$:
            \begin{enumerate}
                \item $\forall x \in A: x \in A \cup B$, because $A \cup B \equiv \{x \in A \lor x \in B\}$
                \item $A \cap (A \cup B) \equiv \{x \in A \land x \in (A \cup B)\}$
                \item As a result, $\forall x \in A: x \in A \cap (A \cup B)$
                \item Also we see $\neg\exists x \in A \cap (A \cup B) : x
                    \notin A$, because $x \in A \cap (A \cup B) \equiv x \in A
                    \land x \in (A \cup B) \equiv x \in A \land (x \in A \lor x
                    \in B)$. This proves $A \cap (A \cup B) = A$ and our
                    hypothesis is correct.
            \end{enumerate}
            Proof $A \cup (A \cap B) = A$:
            \begin{enumerate}
                \item Using the properties of commutativity and associativity we can rearrange the equation like the following:
                \item $A \cup (A \cap B) = (A \cap B) \cup A$
                \item $(A \cap B) \cup A = A \cap (B \cup A)$
                \item $A \cap (B \cup A) = A \cap (A \cup B)$, which was proven in an earlier exercise.
            \end{enumerate}
        \item Let $A, B, X$ be sets such that $A \cup B = X \land A \cap B = \emptyset$. \\
            Proof $A = X \setminus B$:
            \begin{enumerate}
                \item $A \cap B = \emptyset \equiv \forall x \in A \land x \in B: x \in \emptyset$
                \item $\neg\exists x : x \in \emptyset$, so $\forall x \in A: x \notin B$
                \item $X \equiv \{x : x \in A \lor x \in B\}$, and $X \setminus B \equiv \{x : x \in X \land x \notin B\}$, so
                    $X \setminus B \equiv (A \cup B) \setminus B \equiv \{x : (x \in A \lor x \in B) \land x \notin B\}$
                \item Since $\forall x \in X \setminus A: x \in (A \cup B) \land x
                    \notin B$, $X \setminus B = \{x : (x \in A \lor x \in B) \land x \in
                    B\} = \{x: (x \in A \land x \notin B) \lor x (\in B \land x
                    \notin B)\} = \{x: x \in A \land x \notin B\}$
                \item Again, $\forall x \in A: x \notin B$, so $A = {x: x \in A \land x \notin B} = X \setminus B$,
                    and we have proven our hypothesis
            \end{enumerate}
            Proof $B = X \setminus A$
            \begin{enumerate}
                \item The $\cup$ and $\cap$ operators have the property of associativity so we need merely rearrange our
                    starting assumptions to get: \\
                    $A \cup B = B \cup A = X$ and \\
                    $A \cap B = B \cap A = \emptyset$, and invoke our proof from the previous exercise.
            \end{enumerate}
        \item Proof $A \setminus B, B \setminus A$, and $A \cap B$ are disjoint, and that their union is $A \cup B$:
            \begin{enumerate}
                \item First we prove $A \setminus B$ is disjoint from $B
                    \setminus A$ and $A \cap B$, and then that $A \cap B$ is
                    disjoint from $B \setminus A$
                \item $A \setminus B \equiv \{x : x \in A \land x \notin B\}$
                \item $\forall x \in B \setminus A: x \notin A$, so $\forall x
                    \in B \setminus A: x \notin A \setminus B$
                \item Similarly, $\forall x \in A \setminus B: x \notin B$, so
                    $\forall x \in A \setminus B: x \notin B \setminus A$, and
                    we can say that $A \setminus B$ and $B \setminus A$ are
                    disjoint.
                \item By definition, $\forall x \in A \cap B: x \in A \land x \in B$
                \item This means that $\forall x \in A \cap B: x \notin B \setminus A$, because
                    $\forall x \in B \setminus A: x \notin A$.
            \end{enumerate}
            Proof $(A \setminus B) \cup (B \setminus A) \cup (A \cap B) = A \cup B$
            \begin{enumerate}
                \item $(A \setminus B) \cup (B \setminus A) \cup (A \cap B)
                    \equiv \{x : x \in A \land x \notin B\} \cup \{x : x \in B
                    \land x \notin A\} \cup \{x : x \in A \land x \in B\}$
                \item $\{x : x \in A \land x \notin B\} \cup \{x : x \in B
                    \land x \notin A\} = \{x : (x \in A \land x \notin B) \lor
                    (x \in B \land x \notin A) \} = \{x : (x \in A \lor x \in B)
                    \land \neg(x \in A \land x \in B)\}$, thus:
                \item $\{x : x \in A \land x \notin B\} \cup \{x : x \in B
                    \land x \notin A\} \cup \{x : x \in A \land x \in B\} = \{x : (x \in A \lor x \in B)
                    \land \neg(x \in A \land x \in B)\} \cup \{x : x \in A \land x \in B\}$, and:
                \item $\{x : (x \in A \lor x \in B) \land \neg(x \in A \land x
                    \in B)\} \cup \{x : x \in A \land x \in B\} = \{x : x \in A
                    \lor x \in B\} = A \cup B$, finally proving also that $(A \setminus B) \cup (B \setminus A) \cup (A \cap B) = A \cup B$
            \end{enumerate}
        \item Proof axiom of replacement implis axiom of specification:
            \begin{enumerate}
                \item The axiom of replacement states that for any set A and
                    objects $x \in A$ and y ``there exists a set $\{y : P(x, y)$
                    is true for some $x \in A\}$'.
                \item Given this axiom lets consider a possible property $P(x,y) = (y = x \land P(y))$, where
                    $P(x)$ is an arbitrary property.
                \item Applied to the axiom of choice, this property proves the
                    existence of the set $\{y : y = x \land P(y) $ for some $x
                    \in A\}$. 
                \item Applying the substitution property to the objects in
                    question, we can rephrase this set as $\{x: x \in A \land
                    P(x)\} \equiv \{x \in A: P(x)\}$. This is the axiom of 
                    of specification, and so we have derived it as a selective
                    choice of $P(x,y)$.
            \end{enumerate}
    \end{enumerate}
    \subsection{Russell's Paradox}
    \begin{enumerate}
        \item Proof Axiom of Universal Specification implies axiom 3.2
            \begin{enumerate}
                \item There exists a set of elements $\{x : P(x)\}$, for any $P(x)$ valid for all possible objects (Axiom of Universal specification)
                \item Consider the property function $P(x) = False$
                \item This implies the existence of a set $\{x : False\}$, or a set with no elements. (Axiom 3.2)
            \end{enumerate}
            Proof Axiom of Universal Specification implies axiom 3.3
            \begin{enumerate}
                \item There exists a set of elements $\{x : P(x)\}$, for any $P(x)$ valid for all possible objects (Axiom of Universal specification)
                \item Consider the property function $P(x) = (x = y)$ where y is an arbitrary, unique object, either a number or a set itself.
                \item This implies the existence of a set $\{x : P(X) \}$, or $\{x : x = y\}$, the set of that one element. (Axiom 3.3 part 1)
                \item Also consider the property function $P(x) = (x = y \lor x = z)$, where again y and z are arbitrary unique objects.
                \item This implies the existence of a set $\{x : x = y \lor x = z\}$, consisting of the elements y and z, which themselves could be numbers or sets.
            \end{enumerate}
            Proof Axiom of Universal Specification implies axiom 3.4
            \begin{enumerate}
                \item There exists a set of elements $\{x : P(x)\}$, for any $P(x)$ valid for all possible objects (Axiom of Universal specification)
                \item Consider the existence of two arbitrary sets, A and B.
                \item Now consider the property function $P(x) = (x \in A \lor x \in B)$.
                \item This implies the existence of a set $\{x : x \in A \lor x \in B\}$, the definition of the pairwise union function in Axiom 3.4
            \end{enumerate}
            Proof Axiom of Universal Specification implies axiom 3.5
            \begin{enumerate}
                \item There exists a set of elements $\{x : P(x)\}$, for any $P(x)$ valid for all possible objects (Axiom of Universal specification)
                \item Consider the existence of an arbitrary set A.
                \item Now consider the property function $P(x) = (x \in A \land P_2(x))$, where $P_2(x)$ is some other arbitrary function.
                \item The set $\{x : x \in A \land P_2(x)\} \equiv \{x \in A:
                    P_2(x)\}$, and so we can clearly see the Axiom of
                    Specification is a special case of the Axiom of Universal
                    Specification, tied down by the restriction that its
                    property function must be applied to an existing set and
                    not all possible objects.
            \end{enumerate}
            Proof Axiom of Universal Specification implies axiom 3.6
            \begin{enumerate}
                \item There exists a set of elements $\{x : P(x)\}$, for any $P(x)$ valid for all possible objects (Axiom of Universal specification)
                \item Consider the existence of an arbitrary set A.
                \item Now consider the property function $P(y) = (\exists x \in A: P(x,y))$, where $P(x,y)$ is a an arbitrary statement pertaining to x and y, such that
                    for each $x \in A$ there is at most one $y$ where $P(x,y)$ is true
                \item This implies the exxistence of the set $\{y: \exists x \in A: P(x,y)\}$, which is the claim of the axiom of Replacement.
            \end{enumerate}
        \item Proof $A \notin A$
            \begin{enumerate}
                \item Suppose we have a set A.
                \item From the singleton axiom, we can see that there exists the set $\{A\}$ of which A is an element.
                \item From the Axiom of Regularity, we see that $\{A\}$ must contain an element that is either not a set, or is disjoint from $\{A\}$.
                \item Since $\{A\}$ contains only one element --- A --- which
                    is a set --- we also find that A must not contain itself,
                    because it is disjoint from $\{A\}$, and would otherwise
                    violate the Axiom of Regularity.
            \end{enumerate}
            Proof given arbitrary sets $A,B: \neg(A \in B \land B \in A)$
            \begin{enumerate}
                \item Consider arbitrary sets A and B.
                \item Sets do not contain themselves, as established earlier.
                \item Consider now the set $\{A, B\}$.
                \item Due to the Axiom of Regularity, $\{A, B\}$ must contain an element that does not itself contain either A or B.
                \item Neither set contains itself, so A can contain B, or B can contain A, but both sets cannot contain each other at
                    the same time, becuase this would mean $\{A, B\}$ has no elements which do not contain other elements or are not sets.
                    So, $\neg(A \in B \land B \in A)$
            \end{enumerate}
        \item Proof Axiom of Universal Specification implies the existence of the set of all objects.
            \begin{enumerate}
                \item There exists a set of elements $\{x : P(x)\}$, for any proposition $P(x)$ valid for all possible objects (Axiom of Universal specification)
                \item Consider some $P(x) = True$. By Axiom of Comprehension,
                    this shows there is a set such that $\{x : True\}$, or,
                    simplified, the set of all possible objects, because $P(x)$
                    returns true no matter what x is.
            \end{enumerate}
            Proof set of all objects implies Axiom of Universal Specification.
            \begin{enumerate}
                \item Suppose there exists a set of all objects O.
                \item By the Axiom of (Non-Universal) Specification, we know
                    that, given a proposition $P(x)$ vaid for all objects in A,
                    and a set of objects A, there exists some set that contains
                    all the objects in A that satisfy $P(x)$.
                \item This means that we can use the set of all objects along
                    with the Axiom of Specification to declare that there
                    exists a set of elements $\{x \in O: P(x)\}$, which is what
                    is declared by the Axiom of Universal Speification.
            \end{enumerate}
    \end{enumerate}
    \subsection{Functions}
    \begin{enumerate}
        \item Proof the definition of equality in 3.3.7 is reflexive.
            \begin{enumerate}
                \item
                \item Any function $f(x)$ has the same domain and range $X$ and $Y$ of itself.
                \item Any function $f(x)$ has the same output for all $x \in X$, or else we don't consider it a function.
                \item By the definition of equality in the book, this means $f(x) = f(x)$ and so equality between functions is reflexive.
            \end{enumerate}
            Proof equality between functions is symmetric.
            \begin{enumerate}
                \item Suppose we have two functions $f(x)$ and $g(x)$ and $f(x)=g(x)$.
                \item Two functions $f(x)$ and $g(x)$ are equal if and only if
                    they have the same domain, range, and $f(x) = g(x)$ for all
                    x in the domain of these functions. So, the domain of $f(x)$ is
                    the same as the domain of $g(x)$, the range of $f(x)$ is the same
                    as the range of $g(x)$, and $\forall x \in $ the domain of $f(x) : f(x)=g(x)$.
                \item $g(x)$ and $f(x)$ have the same domain and range, because of the 
                    symmetric property of set equality.
                \item $\forall x \in $ the domain of $g(x): g(x) = f(x)$, because of teh symmetric
                    property of elemental equality. So, by our definition earlier, $f(x) = g(x) \implies g(x) = f(x)$.
            \end{enumerate}
            Proof equality between functions is transitive.
            \begin{enumerate}
                \item Two functions $f(x)$ and $g(x)$ are equal if and only if
                    they have the same domain, range, and $f(x) = g(x)$ for all
                    x in the domain of these functions.
                \item Consider three hypothetical functions $f(x)$, $g(x)$ and $h(x)$, where $f(x)=g(x) \land g(x)=h(x)$
                \item $f(x)$ has the same domain as $g(x)$, which is the same domain as $h(x)$. By the substitution property of sets,
                    the domain of $h(x)$ must be equal to the domain of $f(x)$.
                \item Similarly, $f(x)$ has the same range as $g(x)$, which is the same range as $h(x)$. By the substitution property of sets,
                    the range of $h(x)$ must be equal to the range of $f(x)$.
                \item If all three functions have the same domain (hereafter referred to as X), and $\forall x \in X: f(x) = g(x) \land g(x) = h(x)$,
                    then by the substitution property of numbers, $\forall x \in X: f(x) = h(x)$, and we have proven that $f(x) = h(x)$.
            \end{enumerate}
            Proof $(f = g \land h = j) \implies f \circ g = h \circ j$ \\
            TBD
        \item Proof for two functions $f$ and $g$, where the domain of g is
            equal to the range of f, if f and g are injective, then $g \circ f$
            is injective.
            \begin{enumerate}
                \item Consider the domain of f X and range of f Y, where the domain of g is the same as the range of f.
                \item $\forall x \in X: (x \neq x') \implies (f(x) \neq f(x'))$ (Given)
                \item $\forall y \in Y: (y \neq y') \implies (g(y) \neq g(y'))$ (Given)
                \item Consider two unequal $x,x' \in X$. $f(x) \neq f(x')$, as given by the earlier statement.
                \item Since $f(x) \neq f(x')$, and $f(x) \in Y \land f(x') \in
                    Y$, $g(f(x)) \neq g(f(x'))$, as a result of the earlier
                    implication. This is the definition of injectivity, and so the
                    composition $g \circ f$ must be injective.
            \end{enumerate}
            Proof for two functions $f$ and $g$, where the domain of g is
            equal to the range of f, if f and g are surjective, then $g \circ f$
            is surjective.
            \begin{enumerate}
                \item Let's call the domain of f X and range of f Y, where the domain of g is the same as the range of f, and the range of g Z.
                \item $\forall y \in Y: (\exists x \in X: f(x) = y)$ (Given)
                \item $\forall z \in Z: (\exists y \in Y: g(y) = z)$ (Given)
                \item The domain of $g$ is the same as the range of $f$, so there is some value $f(x)$ equal for any value $y \in Y$.
                \item So we can say that $\forall z \in Z: \exists y \in Y: \exists x \in X: z = g(y) = g(f(x))$
                \item This means that for the composition function, $\forall z \in Z: \exists x \in X: z = g(f(x))$, and we have proven the
                    surjectivity of $g \circ f$.
            \end{enumerate}
        \item Not sure the function with an empty set domain makes a whole lot of sense, given previous definitions. Contacted Tao about it. Maybe I'm rarted.
        \item Let $f_1: X \rightarrow Y, f_2: X \rightarrow Y, g_1 Y \rightarrow Z, g_2 \rightarrow Z$. \\
            Show that if $g_1 \circ f_1 = g_1 \circ f_2$ and $g_1$ is injective, that $f_1 = f_2$.
            \begin{enumerate}
                \item $\forall y,y' \in Y: y \neq y' \implies g_1(y) \neq g_1(y')$ (Given)
                \item $\forall y,y' \in Y: g_1(y) = g_2(y') \implies y = y'$ (Contrapositive of above)
                \item $\forall x \in X: g_1(f_1(x)) = g_1(f_2(x))$ (Given)
                \item Thus, $\forall x \in X: f_1(x) = f_2(x)$ (Consequence of previous two statements)
            \end{enumerate}
            Is the same statement true if $g_1$ is not injective? \\
            Not necessarily. Consider the static function $g_1(y) = 7$.
            Since $\forall y \in Y: g_1(y) = 7$, even if $f_1(x) = 6$ and
            $f_2(x) = 23$, $g(f_1(x)) = g(f_2(x))$. So, the same statement
            is not guaranteed to be correct.
            Show that if $g_1 \circ f_1 = g_2 \circ f_1$ and $f_1$ is surjective, that $g_1 = g_2$.
            \begin{enumerate}
                \item $\forall y \in Y: \exists x \in X: f_1(x) = y$ (Given)
                \item Phrased in a different way, the set of all possible outcomes for $f_1(x)$ is the set $Y$.
                \item $\forall x \in X: g_1(f_1(x)) = g_2(f_1(x))$ (Given)
                \item Thus, since there is some $f_1(x)$ for each value in $Y$,
                    $\forall y \in Y: g_1(y) = g_2(y)$, and since $g_1$ and
                    $g_2$ have the same domain and range, $g_1=g_2$.
            \end{enumerate}
            Is the same statement true if $f_1$ is not surjective? \\ 
            Not necessarily.  Consider the static function $f_1(x) = 2$, and
            the functions $g_1(y) = 2y$ and $g_2(y) = y^2$, each with the
            domain and range of all natural numbers. $g_1(f_1(x)) = g_2(f_1(x))
            = 4$, but since $f_1(x)$ only outputs one number on the range of
            the $g$ functions, this does not mean $g_1 = g_2$, and in this case
            they are not equal.
        \item TBD
        \item Let $f: X \rightarrow Y$ be a bijective function, and let
            $f^{-1}: Y \rightarrow X$ be its inverse. \\
            Proof $f^{-1}(f(x)) = x \land f(f^{-1}(y)) = y$ for all $x \in X$
            \begin{enumerate}
                \item $\forall x \in X: (f(x) = y) \implies (f^{-1}(y) = x)$ (Definition of inverse function)
                \item $\forall x \in X: (f^{-1}(f(x)) = x)$ (Substitution)
                \item $\forall x \in X: (f(x) = y) \implies (f(f^{-1}(y)) = y)$ (More substitution)
                \item $\forall y \in Y: \exists x \in X: f(x) = y$ (Given that f is a bijective function)
                \item $\forall y \in Y: f(f^{-1}(y)) = y$ (Combination of last two statements)
            \end{enumerate}
        \item Let $f: X \rightarrow Y$ and $g: Y \rightarrow Z$ be bijective functions. \\
            Prove $g \circ f$ is bijective.
            \begin{enumerate}
                \item $\forall y \in Y: \exists x \in X: f(x) = y$ (Given)
                \item $\forall z \in Z: \exists y \in Y: g(y) = z$ (Given)
                \item $\forall z \in Z: \exists x \in X: g(f(x)) = z$ (From previous two statements, proves surjectivity)
                \item $(x \in X \land x' \in X \land x \neq x') \implies f(x) \neq f(x')$ (Given)
                \item $(y \in Y \land y' \in Y \land y \neq Y') \implies g(y) \neq g(y')$ (Given)
                \item $(x \in X \land x' \in X \land x \neq x') \implies g(f(x)) \neq g(f(x'))$ (From previous two statements, proves injectivity)
            \end{enumerate}
            Prove ${(g \circ f)}^{-1} = g^{-1} \circ f^{-1}$
            \begin{enumerate}
                \item The domain and range of these two function are the same ($Z \rightarrow X$). We must prove them equal for all inputs in their domain (X).
                \item $\forall x \in X: (g \circ f)(x) = z \implies {(g \circ f)}^{-1}(z) = x$ (Definition of inverse function)
                \item $\forall x \in X: {(g \circ f)}^{-1}(f(x)) = x$ (Substitution)
                \item $\forall y \in Y: (g(y) = z) \implies (g^{-1}(z) = y)$ (Definition of inverse function)
                \item $\forall y \in Y: (g^{-1}(g(y)) = y)$ (Substitution)
                \item $\forall x \in X: ((g(f(x)) \equiv (g \circ f)(x)) = z) \implies (g^{-1}(z) = y = f(x))$ (Substitution)
                \item $\forall x \in X: (f(x) = y) \implies (f^{-1}(y) = x)$ (Definition of inverse function)
                \item $\forall x \in X: ((g \circ f)(x) = z) \implies (f^{-1}(g^{-1}(z)) = x)$ (Substitution)
                \item $(g \circ f)$ is bijective, so $\forall z \in Z: \exists x \in X: (g \circ f)(x) = z$, so:
                \item $\forall z \in Z: {(g \circ f)}^{-1}(z) = f^{-1}(g^{-1}(z)) = x$, where $(g \circ f)(x) = z$, and we have proved these function's equality on the domain Z.
            \end{enumerate}
        \item If $X$ is a subset of $Y$, let ${\iota}_{X \rightarrow Y}: X
            \rightarrow Y$ be the inclusion map from $X$ to $Y$, defined by
            mapping $x \mapsto x$ for all $x \in X$, i.e., ${\iota}_{X
            \rightarrow Y}(x) := x$ for all $x \in X$.
            \begin{enumerate}
                \item Show that if $X \subseteq Y \subseteq Z$ then ${\iota}_{Y
                    \rightarrow Z} \circ {\iota}_{X \rightarrow Y} = {\iota}_{X
                    \rightarrow Z}$
                    \begin{enumerate}
                        \item $\forall y \in Y: {\iota}_{Y \rightarrow Z}(y) = y$ (Definition of ${\iota}_{Y \rightarrow Z}$)
                        \item $\forall x \in X: x \in Y$ (Given)
                        \item $\forall x \in X: {\iota}_{Y \rightarrow Z}(x) = x$ (Extension of statement i and ii)
                        \item $\forall x \in X: {\iota}_{X \rightarrow Y}(x) = x$ (Definition of ${\iota}_{X \rightarrow Y}$)
                        \item $\forall x \in X: {\iota}_{Y \rightarrow Z}({\iota}_{X \rightarrow Y}(x)) = x$ (Substitution for statement iii with statement iv)
                        \item $\forall x \in X: {\iota}_{X \rightarrow Z}(x) = x$ (Definition of ${\iota}_{X \rightarrow Z}$)
                        \item $\forall x \in X: {\iota}_{Y \rightarrow
                            Z}({\iota}_{X \rightarrow Y}(x)) =  ({\iota}_{Y
                            \rightarrow Z} \circ {\iota}_{X \rightarrow Y})(x) =
                            {\iota}_{X \rightarrow Z}(x) = x$ (Last two statements)
                        \item The domain and range of of ${\iota}_{X
                            \rightarrow Z}$ are equal to the domain and range
                            of ${\iota}_{X \rightarrow Y} \circ {\iota}_{Y
                            \rightarrow Z}$. (Definition of both functions)
                        \item As a consequence of the last two statements, by
                            definition of functional equality, ${\iota}_{Y
                            \rightarrow Z} \circ {\iota}_{X \rightarrow Y} =
                            {\iota}_{X \rightarrow Z}$
                    \end{enumerate}
                \item Show that if $f : A \rightarrow B$ is any function, then $f = f \circ {\iota}_{A \rightarrow A} = {\iota}_{B \rightarrow B} \circ f$
                    \begin{enumerate}
                        \item $\forall x \in A: {\iota}_{A \rightarrow A}(x) = x$ (Definiton of ${\iota}_{A \rightarrow A}$)
                        \item $\forall x \in B: {\iota}_{B \rightarrow B}(x) = x$ (Definiton of ${\iota}_{A \rightarrow A}$)
                        \item $f$ has a range of $B$. (Given definition of f)
                        \item $\forall x \in A: f(x) = ({\iota}_{B \rightarrow B} \circ f)(x)$ (ii and iii)
                        \item $f$ has a domain of $A$ and ${\iota}_{A \rightarrow A}(x)$ is a bijective function with a domain and range of $A$.
                        \item $\forall x \in A: f(x) = f({\iota}_{A \rightarrow A}(x))$ (i and v)
                        \item $f$ has the same domain and range as $f \circ
                            {\iota}_{A \rightarrow A}$ has the same domain and
                            range as ${\iota}_{B \rightarrow B}$
                        \item By the definitions of functional equality, 
                    \end{enumerate}
                \item Show that if $f : A \rightarrow B$ is a bijective
                    function, then $f \circ f^{-1} = {\iota}_{B \rightarrow
                    B}$ and $f^{-1} \circ f = {\iota}_{A \rightarrow A}$
                    \begin{enumerate}
                        \item $f \circ f^{-1} : B \rightarrow B$ (Definition)
                        \item $\forall y \in B: (f \circ f^{-1})(y) = y$ (Established in earlier proof about the inverse of bijective functions)
                        \item ${\iota}_{B \rightarrow B} : B \rightarrow B$ (Definition)
                        \item $\forall y \in B: {\iota}_{B \rightarrow B}(y) = y$ (Definition)
                        \item $f \circ f^{-1} = {\iota}_{B \rightarrow B}$ (Definition of functional equality)
                        \item $f^{-1} \circ f : A \rightarrow A$ (Definition)
                        \item $\forall x \in A: (f \circ f^{-1})(x) = x$ (Established in earlier proof about the inverse of bijective functions)
                        \item ${\iota}_{A \rightarrow A} : A \rightarrow A$ (Definition)
                        \item $\forall x \in A: {\iota}_{A \rightarrow A}(x) = x$ (Definition)
                        \item $f^{-1} \circ f = {\iota}_{A \rightarrow A}$ (Definition of functional equality)
                    \end{enumerate}
                \item Show that if $X$ and $Y$ are disjoint sets, and $f : X
                    \rightarrow Z$ and $g : Y \rightarrow Z$ are two functions,
                    then there is a unique function $h : X \cup Y \rightarrow
                    Z$ such that $h \circ {\iota}_{X \rightarrow X \cup Y} = f$
                    and $h \circ {\iota}_{Y \rightarrow X \cup Y} = g$ \\
                    Note: Unsure about what ``unique'' means in this context.
                    \begin{enumerate}
                        \item $\forall x \in X: {\iota}_{X \rightarrow X \cup Y}(x) = x$ (Definition)
                        \item Consider some $h: X \cup Y \rightarrow Z$
                        \item $\forall x \in X: (h \circ {\iota}_{X \rightarrow X \cup Y})(x) = h(x)$ (Substitution)
                        \item $\forall y \in Y: {\iota}_{Y \rightarrow X \cup Y}(y) = y$ (Definition)
                        \item $\forall y \in Y: (h \circ {\iota}_{Y \rightarrow X \cup Y})(y) = h(y)$ (Substitution)
                        \item $(x \notin X \land x \notin Y) \implies (x \notin X \cup Y)$ (Proven earlier)
                        \item $\forall x \in X \cup Y: x \in X \oplus x \in Y$ (Given as $X$ is disjoint from $Y$)
                        \item $\forall x \in X \cup Y: (x \in X \implies h(x) = (h \circ {\iota}_{X \rightarrow X \cup Y})(x)) \land (x \in Y \implies h(x) = (h \circ {\iota}_{Y \rightarrow X \cup Y})(x))$
                        \item And taking this further we can say: $\forall
                            x \in X \cup Y: h(x) = (h \circ {\iota}_{X
                            \rightarrow X \cup Y})(x) \lor h(x) = (h \circ
                            {\iota}_{Y \rightarrow X \cup Y})(x)$
                        \item So if $h \circ {\iota}_{X \rightarrow X \cup Y} =
                            f$, and $h \circ {\iota}_{Y \rightarrow X \cup Y} =
                            g$ then we have defined a unique function $h: X
                            \cup Y \rightarrow Z$ 
                    \end{enumerate}
            \end{enumerate}
    \end{enumerate}
    \subsection{Images and Inverse Images}
    \begin{enumerate}
        \item Let $f: X \rightarrow Y$ be a bijective function, and let
            $f^{-1}: Y \rightarrow X$ be its inverse.  Let V be a subset of Y.
            Prove that the forward image of V under $f^{-1}$ is the same set as
            the inverse image of $V$ under $f$.
            \begin{enumerate}
                \item 
            \end{enumerate}
            
    \end{enumerate}
\end{document}
